%%%%%%%%%%%%%%%%%%%%%%%%%%%%%%%%%%%%%%%%%
% "ModernCV" CV and Cover Letter
% LaTeX Template
% Version 1.1 (9/12/12)
%
% This template has been downloaded from:
% http://www.LaTeXTemplates.com
%
% Original author:
% Xavier Danaux (xdanaux@gmail.com)
%
% License:
% CC BY-NC-SA 3.0 (http://creativecommons.org/licenses/by-nc-sa/3.0/)
%
% Important note:
% This template requires the moderncv.cls and .sty files to be in the same
% directory as this .tex file. These files provide the resume style and themes
% used for structuring the document.
%
%%%%%%%%%%%%%%%%%%%%%%%%%%%%%%%%%%%%%%%%%

%----------------------------------------------------------------------------------------
%	PACKAGES AND OTHER DOCUMENT CONFIGURATIONS
%----------------------------------------------------------------------------------------

\documentclass[11pt,a4paper,sans]{moderncv} % Font sizes: 10, 11, or 12; paper sizes: a4paper, letterpaper, a5paper, legalpaper, executivepaper or landscape; font families: sans or roman

\moderncvstyle{classic} % CV theme - options include: 'casual' (default), 'classic', 'oldstyle' and 'banking'
\moderncvcolor{blue} % CV color - options include: 'blue' (default), 'orange', 'green', 'red', 'purple', 'grey' and 'black'
\usepackage[T1]{fontenc}
\usepackage[utf8]{inputenc}
\usepackage{lipsum} % Used for inserting dummy 'Lorem ipsum' text into the template

\usepackage[scale=0.8]{geometry} % Reduce document margins
%\setlength{\hintscolumnwidth}{3cm} % Uncomment to change the width of the dates column
%\setlength{\makecvtitlenamewidth}{10cm} % For the 'classic' style, uncomment to adjust the width of the space allocated to your name

%----------------------------------------------------------------------------------------
%	NAME AND CONTACT INFORMATION SECTION
%----------------------------------------------------------------------------------------

\firstname{Giulio} % Your first name
\familyname{Marin} % Your last name

% All information in this block is optional, comment out any lines you don't need
\title{Curriculum Vitae}

% Anonim
\address{Address on request}{}
\mobile{On request}

% Public
%\address{Vicolo della Vittoria, 16}{31047 Ponte di Piave (TV), Italy}
%\mobile{+39 348 3922696}
%\phone{+1 (650) 944-9483}
%\fax{(000) 111 1113}
\email{giulio.marin@me.com}
\homepage{www.dei.unipd.it/~maringiu}{www.dei.unipd.it/$\sim$maringiu} % The first argument is the url for the clickable link, the second argument is the url displayed in the template - this allows special characters to be displayed such as the tilde in this example
%\extrainfo{additional information}
\photo[110pt][1pt]{pictures/picture} % The first bracket is the picture height, the second is the thickness of the frame around the picture (0pt for no frame)
%\quote{"A witty and playful quotation" - John Smith}

%----------------------------------------------------------------------------------------

\begin{document}

\makecvtitle % Print the CV title

\vspace{-1em}

%----------------------------------------------------------------------------------------
%	PERSONAL SECTION
%----------------------------------------------------------------------------------------

\section{Personal}
\cvitem{First name}{Giulio}
\cvitem{Last name}{Marin}
\cvitem{Gender}{Male}
\cvitem{Date of birth}{September 14, 1989}
\cvitem{Nationality}{Italian}
\cvitem{Visa}{USA H1-B}

%----------------------------------------------------------------------------------------
%	EDUCATION SECTION
%----------------------------------------------------------------------------------------

\section{Education}
\cventry{2014--Present}{Ph.D. student in Information Engineering}{Department of Information Engineering}{University of Padova}{}{Topics: Computer Vision, Image Processing, Machine Learning\\Supervisor: Prof. Pietro Zanuttigh}
\cventry{2011--2013}{M.Sc. ('Laurea Magistrale') in Telecommunication Engineering}{Department of Information Engineering}{University of Padova}{110/110 cum laude}{Dissertation title: ``Confidence Estimation of ToF and Stereo Data for 3D Data Fusion''\\Supervisors: Prof. Pietro Zanuttigh, Ph.D. Carlo Dal Mutto}
\cventry{2008--2011}{B.Sc. ('Laurea Triennale') in Information Engineering}{Department of Information Engineering}{University of Padova}{110/110}{Dissertation title: ``Linear Blend techniques for hand modeling from laser scanner data''\\Supervisors: Prof. Pietro Zanuttigh, Ph.D. Carlo Dal Mutto}

\vspace{1em}
\subsection{Master Thesis}

\cvitem{Title}{\emph{Confidence Estimation of ToF and Stereo Data for 3D Data Fusion}}
\cvitem{Supervisors}{Prof. Pietro Zanuttigh, Ph.D. Carlo Dal Mutto}
\cvitem{Description}{Aim of this thesis is to produce accurate depth maps estimation by combining information from Time-of-Flight (ToF) range cameras and stereo vision systems. These two families of imaging systems, considered alone have complementary characteristics, therefore I combined the two depth information to improve the quality of the acquired three-dimensional scene. Firstly I devised some measures on the correctness of the two depth estimations in order to weight the two hypotheses in the fusion process. Then I extended a technique for cost aggregation to combine the two depth hypothesis according to their reliability. Experimental results show that the proposed fusion approach outperforms the performance of the two systems alone.}
\cvitem{Skills}{\vspace{-12pt}\begin{itemize}
\item Acquisition of 3D data from ToF sensors and depth map computation from stereo vision systems
\item Calibration of heterogeneous imaging systems (ToF and stereo system together)
\item Model the data quality of ToF sensors and stereo systems by estimating confidence measures\end{itemize}}

\subsection{Bachelor Thesis}

\cvitem{Title}{\emph{Linear Blend techniques for hand modeling from laser scanner data}}
\cvitem{Supervisors}{Prof. Pietro Zanuttigh, Ph.D. Carlo Dal Mutto}
\cvitem{Description}{In this work I show how it is possible to create a three-dimensional model of a real hand using a 3D scanner. Through the use of specialized software I also built a skeleton for each model with the aim of create a database containing the hands of different people. These models were used to develop a technique for hand recognition by a 3D camera.}
\cvitem{Skills}{\vspace{-12pt}\begin{itemize}
\item Acquired the knowledge about ICP, Kinematic chains and Linear blend skinning
\item 3D modeling and animation in Autodesk Maya and other 3D mesh processing software
\item Created digital, three dimensional models using a 3D Laser Scanner\end{itemize}}

\vspace{1em}
\subsection{Publications}

% Automatic generated: publications

\vspace{1em}
\subsection{Miscellaneous}

\cvitem{July 2014}{\textbf{International Computer Vision Summer School}}
\cvitem{}{This summer school aimed to provide both an objective and clear overview and an in-depth analysis of the state-of-the-art research in Computer Vision.}


\vspace{1em}
\subsection{Class Projects}

\cvitem{Course}{\textbf{Channel Codes and Capacity 2012-2013}}
\cvitem{Description}{Implementation of encoder and message passing decoder for an LDPC code of 802.11n standard. The code was written in MATLAB with some critical functions in C language to reduce the execution time.}
% \cvitem{Skills}{\vspace{-12pt}\begin{itemize}
% \item Learned to integrate C/C++ function in a MATLAB program
% \item Applied message passing theory for maximum a posteriori symbol detection
% \end{itemize}}

\cvitem{Course}{\textbf{Source Coding 2012-2013}}
\cvitem{Description}{Implementation two different kinds of compression and decompression techniques: arithmetic coding and LZW coding, which is a dictionary based approach used into the GIF compression scheme. The code was written in MATLAB.}
% \cvitem{Skills}{\vspace{-12pt}\begin{itemize}
% \item Workaround to deal with computer's finite precision.
% \item Practical comparison of pros and cons of arithmetic and dictionary based compression techniques with real data (text, image, sound)
% \end{itemize}}

\cvitem{Course}{\textbf{Wireless Systems and Networks 2012-2013}}
\cvitem{Description}{Comparison of Medium Access Control (MAC) techniques for energy harvesting wireless sensor networks (EH-WSN). This project describes the state-of-the-art MAC protocols for EH-WSN and includes performance comparisons among the solutions described.}
% \cvitem{Skills}{\vspace{-12pt}\begin{itemize}
% \item Review the state-of-the-art literature of a topical subject
% \item Write a report based on a group research activity
% \end{itemize}}

\cvitem{Course}{\textbf{Image Processing and 3D Graphics 2011-2012}}
\cvitem{Description}{Designed and implemented a method for the recognition of the boundary of the palm using data from 3D acquisitions. The code was written in MATLAB and C++ with the aid of OpenCV. This was part of a bigger project that had the purpose of tracking the movement of a hand with Microsoft Kinect\textsuperscript{\texttrademark}.}
% \cvitem{Skills}{\vspace{-12pt}\begin{itemize}
% \item Learned to write C++ code and exploit OpenCV algorithms
% \item Read code written by other people and interact with it by making changes and adding instructions.
% \item Used Microsoft Visual Studio and MATLAB Image Processing Toolbox
% \item Solved problems due to external libraries, frameworks and development environment.
% \end{itemize}}

\cvitem{Course}{\textbf{Embedded Systems 2010-2011}}
\cvitem{Description}{('Best App' award) Designed and implemented an Android application able to count, show and store steps recorded during a running or walking session. This application offers support for: a real time plot and other information showed, a list with all sessions stored in the database, a summary, zoomable, and scrollable plot with steps distributed in configurable intervals.}
% \cvitem{Skills}{\vspace{-12pt}\begin{itemize}
% \item Familiarized with computer programming on complex software platforms
% \item Improved the knowledge of Java, SQLite, XML and Eclipse
% \item Program power-constrained devices and design User Interfaces by considering the differences between mobile devices and computers
% \item Share code with Git
% \end{itemize}}

%----------------------------------------------------------------------------------------
%	WORK EXPERIENCE SECTION
%----------------------------------------------------------------------------------------

\section{Work Experience}

\subsection{Vocational}

\cventry{Oct 2014 -- Present}{R\&D Algorithm Engineer}{Aquifi, Inc.}{Palo Alto (CA), U.S.A.}{}{Design and development of low cost and low power active stereo cameras. Development of applications that use 3D information from depth cameras.}

\cventry{Sep 2013 -- Sep 2014}{Consultant}{Aquifi, Inc.}{Palo Alto (CA), U.S.A.}{}{Design, develop and test computer vision algorithms and demos for 3D human/computer interaction. Optimize code for low CPU utilization and low latency. Assist in design, development and testing of application API and SDK using industry best practices. Participate in design and development of other projects as assigned.}

%------------------------------------------------

\cventry{2013\\Jul--Sep}{Intern (R\&D group)}{Imimtek, Inc.}{Sunnyvale (CA), U.S.A.}{}{Worked on computer vision algorithms and demos for 3D human/computer interaction with focus on hand and face tracking.
% \newline{}
% Detailed achievements:
% \begin{itemize}
% \item Developed a C++ framework for object detection and tracking with a stereo vision system, according to the best design patterns practices.
% \item Learned to organize and plan activities for the next two weeks, according to the terms established by the company
% \item Present the work done every two weeks during company meetings, writing technical reports and showing practical demos.
% \item Study state-of-the-art papers and solutions to present to other colleagues during 'Technical Updates' meetings on a weekly basis.\newline{}
% \end{itemize}
}

%------------------------------------------------
\vspace{1em}
\subsection{Miscellaneous}

\cventry{2008--2011 (Summer job)}{Worker cellarman}{Cantina Viticoltori Ponte di Piave S.n.c.}{Ponte di Piave (TV), Italy}{}{Operations and activities relating to the collection and vinification of grapes delivered by members.
% \newline{}\newline{}
% Detailed achievements:
% \begin{itemize}
% \item Learned how to work and relate with coworkers
% \item Organize activity to other people\newline{}
% \end{itemize}
}

%------------------------------------------------

\cventry{2006--2007 (Summer job)}{Electrician}{ELECTRO NOVA DI FELET A. \& PARRO G. SNC}{Oderzo (TV), Italy}{}{Creation and maintenance of electrical, civil and industrial automation in general (electric panels, alarm systems) home automation and TVCC.
% \newline{}\newline{}
% Detailed achievements:
% \begin{itemize}
% \item Improvement in SIEMENS PLC programming
% \item Design and realize electrical circuits for ad-hoc mechanical machines
% \end{itemize}
}

%------------------------------------------------

\cventry{June 2007}{Intern}{Department of Molecular Sciences and Nanosystems}{University of Venice}{}{Winner of 'Progetto Lauree Scientifiche' (promoted by MIUR, Confindustria).\newline{}Developed a small research project on 'inverse opals'.}

%----------------------------------------------------------------------------------------
%	AWARDS SECTION
%----------------------------------------------------------------------------------------

\section{Awards}

\cvitem{Jun 2011}{'Best App' award on Embedded Systems course\newline{}\textit{Department of Information Engineering}, University of Padova}
\cvitem{Nov 2007}{Winner of National Contest of Electrotechnic\newline{}\textit{ITIS Galileo Ferraris}, Molfetta (BA)}

%----------------------------------------------------------------------------------------
%	COMPUTER SKILLS SECTION
%----------------------------------------------------------------------------------------

\section{Computer skills}

\cvitem{Advanced}{\textsc{matlab}, C++, OpenCV, \LaTeX, Git, Computer Hardware and Support, Microsoft Windows, MAC OS, Microsoft Visual Studio, XCode}
\cvitem{Intermediate}{\textsc{java}, C, OpenGL, Eclipse, Design Patterns, \textsc{html/css}, Linux, Autodesk AutoCAD}
\cvitem{Basic}{Objective-C, Autodesk Maya}
\cvitem{Others}{ECDL}

%----------------------------------------------------------------------------------------
%	Social SKILLS SECTION
%----------------------------------------------------------------------------------------

\section{Social skills and Competences}

\cvitem{Team work}{I have worked in different kind of teams from research group to sport crew.}
\cvitem{Manual skills}{I realized different electro-mechanical structure. For example, for the Secondary School final project, I made a 'Solar Concentrator', that was brought to various exhibitions and conferences. Furthermore, for the Bachelor thesis I made a rotating support for a '3D laser scanner'.}
\cvitem{Organization}{During my studies I supervised students working on research projects, providing support and checking the progress of their work organizing weekly meetings. In my work experiences I analyzed problems proposing and implementing possible solutions.}
\cvitem{Driving license}{Italy: categories A1 and B. California: category C.}
\cvitem{Memberships}{IEEE Student Member}

%----------------------------------------------------------------------------------------
%	LANGUAGES SECTION
%----------------------------------------------------------------------------------------

\section{Languages}

\cvitem{Italian}{\textbf{Mother tongue}}
\cvitem{English}{\textbf{Intermediate (C1)}\newline{}Conversationally fluent.}
\cvitem{French}{\textbf{Basic}\newline{}Basic words and phrases only. Middle School level.}

%----------------------------------------------------------------------------------------
%	INTERESTS SECTION
%----------------------------------------------------------------------------------------

\section{Interests}

\cvitem{Academic}{I have attended some online courses (Coursera and edX platforms) not only on computer science topics, but also on economics and business management. During last years I've also held private lessons to undergraduate students with main topics Maths and Physics.}
\cvitem{Sport}{I played soccer at a competitive level for about 13 years. Now I practice sport regularly like running, mountain biking, hiking, rock climbing and skiing.}

%----------------------------------------------------------------------------------------
%	COVER LETTER
%----------------------------------------------------------------------------------------

% To remove the cover letter, comment out this entire block

%\clearpage
%
%\recipient{HR Departmnet}{Corporation\\123 Pleasant Lane\\12345 City, State} % Letter recipient
%\date{\today} % Letter date
%\opening{Dear Sir or Madam,} % Opening greeting
%\closing{Sincerely yours,} % Closing phrase
%\enclosure[Attached]{curriculum vit\ae{}} % List of enclosed documents
%
%\makelettertitle % Print letter title
%
%\lipsum[1-3] % Dummy text
%
%\makeletterclosing % Print letter signature
%
%----------------------------------------------------------------------------------------
\vfill
\begin{flushright}
\textit{Palo Alto, \today}
\end{flushright}

\end{document}
