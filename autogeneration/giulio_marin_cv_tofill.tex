%%%%%%%%%%%%%%%%%%%%%%%%%%%%%%%%%%%%%%%%%
% "ModernCV" CV and Cover Letter
% LaTeX Template
% Version 1.1 (9/12/12)
%
% This template has been downloaded from:
% http://www.LaTeXTemplates.com
%
% Original author:
% Xavier Danaux (xdanaux@gmail.com)
%
% License:
% CC BY-NC-SA 3.0 (http://creativecommons.org/licenses/by-nc-sa/3.0/)
%
% Important note:
% This template requires the moderncv.cls and .sty files to be in the same
% directory as this .tex file. These files provide the resume style and themes
% used for structuring the document.
%
%%%%%%%%%%%%%%%%%%%%%%%%%%%%%%%%%%%%%%%%%

%----------------------------------------------------------------------------------------
%	PACKAGES AND OTHER DOCUMENT CONFIGURATIONS
%----------------------------------------------------------------------------------------

\documentclass[11pt,a4paper,sans]{moderncv} % Font sizes: 10, 11, or 12; paper sizes: a4paper, letterpaper, a5paper, legalpaper, executivepaper or landscape; font families: sans or roman

\moderncvstyle{classic} % CV theme - options include: 'casual' (default), 'classic', 'oldstyle' and 'banking'
\moderncvcolor{blue} % CV color - options include: 'blue' (default), 'orange', 'green', 'red', 'purple', 'grey' and 'black'
\usepackage[T1]{fontenc}
\usepackage[utf8]{inputenc}
\usepackage{lipsum} % Used for inserting dummy 'Lorem ipsum' text into the template

\usepackage[scale=0.9]{geometry} % Reduce document margins
%\setlength{\hintscolumnwidth}{3cm} % Uncomment to change the width of the dates column
%\setlength{\makecvtitlenamewidth}{10cm} % For the 'classic' style, uncomment to adjust the width of the space allocated to your name

\usepackage{changepage}
\newenvironment{publications}
{
\begin{adjustwidth}{\hintscolumnwidth + \separatorcolumnwidth}{}
\vspace{-15pt}
\begin{itemize}
}
{
\end{itemize}
\end{adjustwidth}
}

%----------------------------------------------------------------------------------------
%	NAME AND CONTACT INFORMATION SECTION
%----------------------------------------------------------------------------------------

\firstname{Giulio} % Your first name
\familyname{Marin} % Your last name

% All information in this block is optional, comment out any lines you don't need
\title{R\&D Computer Vision Engineer}

% Anonim
%\address{573 Montego Ter}{Sunnyvale, CA 94089}
%\mobile{+1 (650) 944-9483}

% Public
%\address{Vicolo della Vittoria, 16}{31047 Ponte di Piave (TV), Italy}
%\mobile{+39 348 3922696}
%\phone{+1 (650) 944-9483}
%\fax{(000) 111 1113}
%\email{giulio.marin@me.com}
%\homepage{giuliomarin.github.io}{giuliomarin.github.io} % The first argument is the url for the clickable link, the second argument is the url displayed in the template - this allows special characters to be displayed such as the tilde in this example
%\extrainfo{additional information}
%\photo[100pt][1pt]{pictures/picture} % The first bracket is the picture height, the second is the thickness of the frame around the picture (0pt for no frame)
%\quote{"A witty and playful quotation" - John Smith}

%----------------------------------------------------------------------------------------

\begin{document}

\makecvtitle % Print the CV title

\vspace{-8em}

%----------------------------------------------------------------------------------------
%	PERSONAL SECTION
%----------------------------------------------------------------------------------------

\section{Personal}
%\cvitem{First name}{Giulio}
%\cvitem{Last name}{Marin}
%\cvitem{Gender}{Male}
%\cvitem{Date of birth}{September 14, 1989}
\cvitem{Contacts}{\href{mailto:giulio.marin@me.com}{giulio.marin@me.com}}
%\cvitem{Contacts}{\href{mailto:giulio.marin@me.com}{giulio.marin@me.com} - +1 (650) 944-9483}
%\cvitem{Address}{910 Lenora St, Seattle, WA 98121}
\cvitem{Nationality}{Italian, USA H1-B visa}
\cvitem{Website}{\href{http://giuliomarin.github.io}{giuliomarin.github.io}}

%----------------------------------------------------------------------------------------
%	WORK EXPERIENCE SECTION
%----------------------------------------------------------------------------------------

\section{Work Experience}

\subsection{Vocational}

\cventry{Oct 2014 -- Present}{R\&D Computer Vision Engineer}{Aquifi, Inc.}{Palo Alto, CA}{}{Member of the R\&D group of 4 people that designed low cost and low power active stereo cameras that later became the core product of the company and led to Series C financing. Developed applications with engineering group using 3D information from color and depth cameras for 3D reconstruction, AR/VR, manufacturing and logistics. Optimized algorithms to run efficiently on embedded systems with emphasis on speed and accuracy. Demoed products and prototypes to potential customers.}

\cventry{Sep 2013 -- Sep 2014}{Consultant}{Aquifi, Inc.}{Palo Alto, CA}{}{Designed and developed computer vision algorithms and demos for touchless human/computer interaction with applications in remote interaction and privacy monitoring. Optimized code for low CPU utilization and low latency. Assisted in design, development and testing of application API and SDK using industry best practices. Implemented state of the art computer vision and machine learning algorithms.}

\cventry{2013\\Jul--Sep}{Intern (R\&D group)}{Imimtek, Inc. (now Aquifi Inc.)}{Sunnyvale, CA}{}{Worked in a team with other computer vision PHDs to define the company's future products. Produced design that was then implemented in engineering. Developed algorithms and demos for hand and face tracking with low cost vision sensors.
% \newline{}
% Detailed achievements:
% \begin{itemize}
% \item Developed a C++ framework for object detection and tracking with a stereo vision system, according to the best design patterns practices.
% \item Learned to organize and plan activities for the next two weeks, according to the terms established by the company
% \item Present the work done every two weeks during company meetings, writing technical reports and showing practical demos.
% \item Study state-of-the-art papers and solutions to present to other colleagues during 'Technical Updates' meetings on a weekly basis.\newline{}
% \end{itemize}
}

%%------------------------------------------------
%\vspace{1em}
%\subsection{Miscellaneous}
%
%\cventry{2008--2011 (Summer job)}{Worker cellarman}{Cantina Viticoltori Ponte di Piave S.n.c.}{Ponte di Piave (TV), Italy}{}{Operations and activities relating to the collection and vinification of grapes delivered by members.
%% \newline{}\newline{}
%% Detailed achievements:
%% \begin{itemize}
%% \item Learned how to work and relate with coworkers
%% \item Organize activity to other people\newline{}
%% \end{itemize}
%}
%
%%------------------------------------------------
%
%\cventry{2006--2007 (Summer job)}{Electrician}{ELECTRO NOVA DI FELET A. \& PARRO G. SNC}{Oderzo (TV), Italy}{}{Creation and maintenance of electrical, civil and industrial automation in general (electric panels, alarm systems) home automation and TVCC.
%% \newline{}\newline{}
%% Detailed achievements:
%% \begin{itemize}
%% \item Improvement in SIEMENS PLC programming
%% \item Design and realize electrical circuits for ad-hoc mechanical machines
%% \end{itemize}
%}
%
%%------------------------------------------------
%
%\cventry{June 2007}{Intern}{Department of Molecular Sciences and Nanosystems}{University of Venice}{}{Winner of 'Progetto Lauree Scientifiche' (promoted by MIUR, Confindustria).\newline{}Developed a small research project on 'inverse opals'.}
%

%----------------------------------------------------------------------------------------
%	EDUCATION SECTION
%----------------------------------------------------------------------------------------

\section{Education}
\cventry{2014--2017	}{Ph.D. in Information Engineering}{}{University of Padova, Italy.}{}{Thesis: \emph{3D data fusion from multiple sensors and its applications}.\\Advisor: Prof.~Pietro~Zanuttigh\\[0.5em]
My research is in computer vision and machine learning, in particular in the acquisition and processing of 3D scenes acquired from multiple sensors. My thesis focuses on combining color and depth information to improve the quality of the data independently produced by the sensors. I also worked on gesture recognition from RGB-D data and other sensors.}
\cventry{2011--2013}{M.Sc. in Telecommunication Engineering}{}{University of Padova, Italy.}{}{Thesis: \emph{Confidence Estimation of ToF and Stereo Data for 3D Data Fusion}.\\Advisor: Prof.~Pietro~Zanuttigh\\
Grade: 110/110 cum laude.}
\cventry{2008--2011}{B.Sc. in Information Engineering}{}{University of Padova, Italy.}{}{Thesis: \emph{Linear Blend techniques for hand modeling from laser scanner data}.\\Advisor: Prof.~Pietro~Zanuttigh\\
``Best Android App'' award for Embedded Systems class.}
%
%\vspace{1em}
%\subsection{Master Thesis}
%
%\cvitem{Title}{\emph{Confidence Estimation of ToF and Stereo Data for 3D Data Fusion}}
%\cvitem{Supervisors}{Prof. Pietro Zanuttigh, Ph.D. Carlo Dal Mutto}
%\cvitem{Description}{Aim of this thesis is to produce accurate depth maps estimation by combining information from Time-of-Flight (ToF) range cameras and stereo vision systems. These two families of imaging systems, considered alone have complementary characteristics, therefore I combined the two depth information to improve the quality of the acquired three-dimensional scene. Firstly I devised some measures on the correctness of the two depth estimations in order to weight the two hypotheses in the fusion process. Then I extended a technique for cost aggregation to combine the two depth hypothesis according to their reliability. Experimental results show that the proposed fusion approach outperforms the performance of the two systems alone.}
%\cvitem{Skills}{\vspace{-12pt}\begin{itemize}
%\item Acquisition of 3D data from ToF sensors and depth map computation from stereo vision systems
%\item Calibration of heterogeneous imaging systems (ToF and stereo system together)
%\item Model the data quality of ToF sensors and stereo systems by estimating confidence measures\end{itemize}}
%
%\subsection{Bachelor Thesis}
%
%\cvitem{Title}{\emph{Linear Blend techniques for hand modeling from laser scanner data}}
%\cvitem{Supervisors}{Prof. Pietro Zanuttigh, Ph.D. Carlo Dal Mutto}
%\cvitem{Description}{In this work I show how it is possible to create a three-dimensional model of a real hand using a 3D scanner. Through the use of specialized software I also built a skeleton for each model with the aim of create a database containing the hands of different people. These models were used to develop a technique for hand recognition by a 3D camera.}
%\cvitem{Skills}{\vspace{-12pt}\begin{itemize}
%\item Acquired the knowledge about ICP, Kinematic chains and Linear blend skinning
%\item 3D modeling and animation in Autodesk Maya and other 3D mesh processing software
%\item Created digital, three dimensional models using a 3D Laser Scanner\end{itemize}}
%

%\vspace{1em}
%\subsection{Miscellaneous}
%
%\cvitem{July 2014}{\textbf{International Computer Vision Summer School}}
%\cvitem{}{This summer school aimed to provide both an objective and clear overview and an in-depth analysis of the state-of-the-art research in Computer Vision.}

%
%\vspace{1em}
%\subsection{Class Projects}
%
%\cvitem{Course}{\textbf{Channel Codes and Capacity 2012-2013}}
%\cvitem{Description}{Implementation of encoder and message passing decoder for an LDPC code of 802.11n standard. The code was written in MATLAB with some critical functions in C language to reduce the execution time.}
%% \cvitem{Skills}{\vspace{-12pt}\begin{itemize}
%% \item Learned to integrate C/C++ function in a MATLAB program
%% \item Applied message passing theory for maximum a posteriori symbol detection
%% \end{itemize}}
%
%\cvitem{Course}{\textbf{Source Coding 2012-2013}}
%\cvitem{Description}{Implementation two different kinds of compression and decompression techniques: arithmetic coding and LZW coding, which is a dictionary based approach used into the GIF compression scheme. The code was written in MATLAB.}
%% \cvitem{Skills}{\vspace{-12pt}\begin{itemize}
%% \item Workaround to deal with computer's finite precision.
%% \item Practical comparison of pros and cons of arithmetic and dictionary based compression techniques with real data (text, image, sound)
%% \end{itemize}}
%
%\cvitem{Course}{\textbf{Wireless Systems and Networks 2012-2013}}
%\cvitem{Description}{Comparison of Medium Access Control (MAC) techniques for energy harvesting wireless sensor networks (EH-WSN). This project describes the state-of-the-art MAC protocols for EH-WSN and includes performance comparisons among the solutions described.}
%% \cvitem{Skills}{\vspace{-12pt}\begin{itemize}
%% \item Review the state-of-the-art literature of a topical subject
%% \item Write a report based on a group research activity
%% \end{itemize}}
%
%\cvitem{Course}{\textbf{Image Processing and 3D Graphics 2011-2012}}
%\cvitem{Description}{Designed and implemented a method for the recognition of the boundary of the palm using data from 3D acquisitions. The code was written in MATLAB and C++ with the aid of OpenCV. This was part of a bigger project that had the purpose of tracking the movement of a hand with Microsoft Kinect\textsuperscript{\texttrademark}.}
%% \cvitem{Skills}{\vspace{-12pt}\begin{itemize}
%% \item Learned to write C++ code and exploit OpenCV algorithms
%% \item Read code written by other people and interact with it by making changes and adding instructions.
%% \item Used Microsoft Visual Studio and MATLAB Image Processing Toolbox
%% \item Solved problems due to external libraries, frameworks and development environment.
%% \end{itemize}}
%
%\cvitem{Course}{\textbf{Embedded Systems 2010-2011}}
%\cvitem{Description}{('Best App' award) Designed and implemented an Android application able to count, show and store steps recorded during a running or walking session. This application offers support for: a real time plot and other information showed, a list with all sessions stored in the database, a summary, zoomable, and scrollable plot with steps distributed in configurable intervals.}
%% \cvitem{Skills}{\vspace{-12pt}\begin{itemize}
%% \item Familiarized with computer programming on complex software platforms
%% \item Improved the knowledge of Java, SQLite, XML and Eclipse
%% \item Program power-constrained devices and design User Interfaces by considering the differences between mobile devices and computers
%% \item Share code with Git
%% \end{itemize}}

%----------------------------------------------------------------------------------------
%	TECHNICAL SKILLS SECTION
%----------------------------------------------------------------------------------------

\section{Technical skills}

\cvitem{Advanced}{C++, Matlab, Python, OpenCV, Git, XCode, Microsoft Visual Studio}
\cvitem{Intermediate}{Java, C, OpenGL, SIMD, Eclipse, HTML/CSS, Linux, Autodesk AutoCAD}
\cvitem{Basic}{Objective-C, Autodesk Maya}


%%----------------------------------------------------------------------------------------
%%	AWARDS SECTION
%%----------------------------------------------------------------------------------------
%
%\section{Awards}
%
%\cvitem{Jun 2011}{'Best Android App' award on Embedded Systems course\newline{}\textit{Department of Information Engineering}, University of Padova}
%\cvitem{Nov 2007}{Winner of National Contest of Electrotechnic\newline{}\textit{ITIS Galileo Ferraris}, Molfetta (BA)}

%----------------------------------------------------------------------------------------
%	Social SKILLS SECTION
%----------------------------------------------------------------------------------------

\section{Other skills}

\cvitem{Team player}{My experience in team dynamics ranges from fast-paced startups to leading competitive soccer teams. During my PHD, I worked closely with faculty to share research best practices with international visitors. I enjoy mentoring other students and deliver a series of lectures each year in computer vision to Master's candidates.}
\cvitem{Project management}{I like to work on challenging problems, propose and quickly prototype possible solutions, as well as product oriented longer term projects. I organize projects with parallel workstreams with an emphasis on building scalable code. During my PHD I supervised 10 students on their thesis and research projects, providing coaching on how to define problems and efficiently plan actual solutions.}
%\cvitem{Driving license}{Italy: categories A1 and B. California: category C.}
%\cvitem{Memberships}{IEEE Student Member}

%----------------------------------------------------------------------------------------
%	INTERESTS SECTION
%----------------------------------------------------------------------------------------

\section{Interests}

\cvitem{Academic}{I'm a reviewer and TPC member for research publications ICIP, ICME and IEEE Signal Processing Letters. I also follow online courses on economics and business management. In the past I've tutored undergraduate students in math and physics.}
\cvitem{Projects}{I enjoy developing electro-mechanical devices and have built a solar concentrator and a rotating support for a 3D laser scanner.}
\cvitem{Sports}{I played competitive soccer in an Italian youth league for 12 years. Now I regularly run, mountain bike, hike, rock climb and ski.}

%----------------------------------------------------------------------------------------
%	LANGUAGES SECTION
%----------------------------------------------------------------------------------------

\section{Languages}

\cvitem{English}{Fluent}
\cvitem{Italian}{Native speaker}
%\cvitem{French}{\textbf{Basic} Basic words and phrases only. Middle School level.}

%----------------------------------------------------------------------------------------
%	PUBLICATIONS
%----------------------------------------------------------------------------------------

%\vspace{1em}
\section{Publications}

% Automatic generated: publications

%----------------------------------------------------------------------------------------
%	COVER LETTER
%----------------------------------------------------------------------------------------

% To remove the cover letter, comment out this entire block

%\clearpage
%
%\recipient{HR Departmnet}{Corporation\\123 Pleasant Lane\\12345 City, State} % Letter recipient
%\date{\today} % Letter date
%\opening{Dear Sir or Madam,} % Opening greeting
%\closing{Sincerely yours,} % Closing phrase
%\enclosure[Attached]{curriculum vit\ae{}} % List of enclosed documents
%
%\makelettertitle % Print letter title
%
%\lipsum[1-3] % Dummy text
%
%\makeletterclosing % Print letter signature
%
%----------------------------------------------------------------------------------------
\vspace{1em}
\begin{flushright}
\textit{Palo Alto, \today}
\end{flushright}

\end{document}
